\documentclass[a4paper,10pt]{report}
\usepackage[utf8]{inputenc}
\usepackage[T1]{fontenc}
\usepackage{polyglossia}
\defaultfontfeatures{Ligatures={TeX},Renderer=Basic}
\setmainfont[Ligatures={TeX,Historic}]{Myriad Pro}
\setdefaultlanguage[spelling=modern]{russian}

\usepackage{HSE-theme/beamerthemeHSE}

\newcommand{\ldf}{"Локализатор .desktop-файлов"}
\newcommand{\df}{.desktop-файл}
\newcommand{\rosalab}{ООО "НТЦ ИТ РОСА"}
\newcommand{\rosa}{"РОСА"}
\newcommand{\rosalinux}{ROSA Linux}
\newcommand{\ieee}{IEEE-830-1998}
\newcommand{\fedora}{Fedora Workstation 22}

\title{Требования к программному обеспечению \\ Программа для автоматизации труда локализатора пакетов \rosalinux \\ \ldf}
\author{Громов Евгений \\ Яковлев Дмитрий \\ Ериков Михаил}

\begin{document}

\maketitle

\begin{abstract}
В этом документе описываются требования к программе-локализатору {\df}а, которая пишется для {\rosalab} и их {\rosalinux} в рамках курса "Командный проект по программной инженерии". Документ оформлялся в соответствии с рекомендациями стандарта \ieee.
\end{abstract}

\tableofcontents

\chapter{Введение}	
\section{Назначение}
Этот документ следует использовать при проектировании \ldf. Целевой аудиторией данного документа являются разработчики программ и представители компании \rosa.

\section{Область действия}
Данные требования распространяются на программный продукт \ldf. \ldf создаётся для автоматизации труда переводчика. \ldf обнаруживает расположение {\df}ов и предлагает опции для перевода.

\section{Определения, акронимы и сокращения}
\begin{enumerate}
	\item ABF --- (Automatic Build Farm) автоматизированная сборочная система, выполняющая функции хостинга исходных кодов и непрерывной сборки.
	\item RPM --- (RPM Package Manager) формат хранения пакетов в Linux.
	\item API --- (application programming interface)  интерфейс прикладного программирования.
\end{enumerate}

\section{Публикации}
Этот подраздел должен:
\begin{enumerate}
	\item Представить полный список всех документов, на которые делаются ссылки в других местах SRS;
	\item Идентифицировать каждый документ по заголовку, номеру отчета (если применяется), дате и издательской организации;
	\item Определить источники, из которых могут быть получены ссылки.
\end{enumerate}	
Эту информацию можно обеспечить ссылкой на приложение или другой документ.

\section{Краткий обзор}
Далее будут предоставлены общие описания производимого программного продукта и список требований к нему. Требования включают в себя следующие части:
\begin{itemize}
	\item "Общие описания": даётся общее объяснение программного продукта, достаточное для поверхностного понимания объёма работ и состава программного продукта;
	\item "Специфические требования": предоставлено детальное описание требований к программе и её функций.
\end{itemize}

\chapter{Общее описание}

\section{Перспектива изделия}
\ldf будет использоваться в процессе добавления и обновления пакетов в дистрибутиве ROSA Linux. Система должна взаимодействовать с ABF как с репозиторием исходных кодов и хранилищем информации о процессе сборки приложения.

\subsubsection{Интерфейсы пользователя}
Программа будет использоваться в графическом режиме при разрешении экрана, равном или превышающим 1024 на 768 точек.

\subsection{Интерфейсы программного обеспечения}
Программа должна использовать API Git, ABF.


\section{Функции изделия}
\subsection{Локализация {\df}ов}
При использовании программы для локализации {\df}ов программа получает из поданных на вход файлов RPM информацию о расположении локализуемого файла и предоставляет возможность в ручном или автоматическом режиме выполнить его локализацию.\textbf{}

\subsection{Характеристики пользователя}
Пользователем системы является сотрудник \rosalab. Пользователь имеет представление о местах расположения локализуемых файлов и правилах локализации приложений.

\subsection{Ограничения}
Ограничения не накладываются.

\subsection{Допущения и зависимости}
Предполагается, что пользователь программы 	{\ldf} работает с ней на операционной системе {\fedora} или \rosalinux. Предполагается, что система установлена в варианте по-умолчанию на совместимом с этими операционными системами аппаратном или виртуализованном обеспечении.

\subsection{Распределение требований}
В будущих версиях системы может быть реализована работа с другими поставщиками информации о RPM-пакетах, а также с иными источниками переведённых строк.

\chapter{Специфические требования}

\section{Внешние интерфейсы}
Программа дожна иметь графический интерфейс, написанный с использованием библиотеки QT.
Требования к графическому интерфейсу:
\begin{enumerate}
	\item \ldf должен иметь стартовый экран со списком импортированных RPM и их статусами;
	\item \ldf должен иметь функцию импорта нескольких RPM и добавления их в список отслеживания;
	\item интерфейс должен иметь функцию выбора рабочей ветки системы контроля версий Git;
	\item \ldf должен отображать статус каждого импортированного пакета:
	\begin{enumerate}
		\item импортирован
		\item получена информация об источниках исходных кодов
		\item ошибка получения информации об источнике исходного кода
		\item определён источник {\df}а
		\item не обнаружен \df
		\item {\df} локализован в отслеживаемой ветке исходного кода
		\item проведена машинная локализация
		\item ошибка проведения машинной локализации
		\item предоставлен пользовательский перевод
		\item текущая версия перевода включена в систему контроля версий исходного кода
	\end{enumerate}
	\item \ldf должен иметь экран операций над одним выбранным пакетом.
\end{enumerate}

\section{Функции}
\begin{enumerate}
	\item Программа должна получать информацию о расположении исходных кодов приложения, варианте поставки {\df}а и статусе применения изменений из предоставленного пользователем RPM.
	\item Программа должна осуществлять машинный перевод извлечённых из {\df}а строк и давать возможность пользователю изменить автоматически осуществлённый перевод.
	\item Программа должна формировать список изменений в виде коммита в системе контроля версий Git и отправлять такие изменения в хранилище.	
\end{enumerate}

\section{Требования к рабочим характеристикам}
\begin{enumerate}
	\item Программа должна позволять пользователю осуществлять локализацию одного пакета RPM за раз.
	\item Программа должна предпринимать попытки получения данных об RPM и машинного перевода в фоновом режиме.
\end{enumerate}



\section{Логические требования к базе данных}
Программа должна хранить информацию об известных ей пакетах в домашней директории пользователя.

\section{Проектные ограничения}
Ограничения не налагаются

\section{Атрибуты системы программного обеспечения}
\subsection{Надёжность}
Программа должна запрашивать подтверждение введённых данных перед отправкой изменений в репозиторий исходного кода.

\subsection{Удобство сопровождения}
Программа должна быть написана с использованием скриптового языка программирования Python.

\subsection{Мобильность}
Программа не должна содержать архитектурно-зависимых компонентов или кода.

\end{document}
